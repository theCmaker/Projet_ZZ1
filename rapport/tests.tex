Nous présentons dans cette partie un ensemble de tests que nous avons effectués pour contrôler le bon fonctionnement de nos algorithmes. 

Nous avons réalisé des graphiques afin d'avoir une meilleure interprétation des résultats obtenus.

Sur ces graphiques, nous avons choisi de représenter les mobiles $M_i$ non-interceptés par des croix vertes (\tikz[baseline=-0.5ex]{\node[mobile,inner sep=0,outer sep=0]{\mobile};}), et les mobiles $M_i$ interceptés par des croix rouges (\tikz[baseline=-0.5ex]{\node[caught,inner sep=0,outer sep=0]{\mobile};}).

Les vecteurs vitesse et la trajectoire empruntée par les mobiles sont indiqués en vert par des vecteurs (\tikz[baseline=-0.5ex]{\draw[speed] (0,0) -- (1,0);}) et des lignes pointillées (\tikz[baseline=-0.5ex]{\draw[camino] (0,0) -- (1,0);}).

La position initiale de l'intercepteur est repérée par un carré bleu (\tikz[baseline=-0.5ex]{\node[interceptor,inner sep=0,outer sep=0]{\interceptor};}) et ses positions successives par des croix bleues (\tikz[baseline=-0.5ex]{\node[interceptor,inner sep=0,outer sep=0]{\mobile};}). La date de la dernière interception est indiquée au-dessus de la position où elle a lieu.

\section{Test \no1: Tous les mobiles interceptés, séquences différentes}
  \begin{listing}[H]
    \textfile{../tests/test_1/test_1.data}
    \caption{test\_1.data}
  \end{listing}

  \begin{figure}[H]
    \begin{center}
      \boxed{
      \begin{tikzpicture}[scale=0.7]
        \draw[grided,step=1.0,thin] (-3.000000,-2.000000) grid (14.344612,8.703384);
\draw (-3.000000,0) -- coordinate (x axis mid) (14.344612,0);
\draw (0,-2.000000) -- coordinate (y axis mid) (0,8.703384);
\foreach \x in {-3,...,14}
  \draw (\x,1pt) -- (\x,-3pt) node[anchor=north] {\x};
\foreach \y in {-2,...,8}
  \draw (1pt,\y) -- (-3pt,\y) node[anchor=east] {\y};
\node[interceptor] (I0) at (0.000000,0.000000) {\interceptor};
\node[mobile,anchor=center] (M0) at (1.000000,2.000000) {\mobile};
\node[mobile] at (M0.south east) {$M_0$};
\draw[speed] (M0.center) -- ($ (M0.center) + (-1.000000,0.000000) $);
\node[mobile,anchor=center] (M1) at (0.000000,3.000000) {\mobile};
\node[mobile] at (M1.south east) {$M_1$};
\draw[speed] (M1.center) -- ($ (M1.center) + (0.000000,-1.000000) $);
\node[mobile,anchor=center] (M2) at (2.000000,5.000000) {\mobile};
\node[mobile] at (M2.south east) {$M_2$};
\draw[speed] (M2.center) -- ($ (M2.center) + (1.000000,0.300000) $);
\node[mobile,anchor=center] (M3) at (4.000000,1.000000) {\mobile};
\node[mobile] at (M3.south east) {$M_3$};
\draw[speed] (M3.center) -- ($ (M3.center) + (0.500000,0.500000) $);
\node[mobile,anchor=center] (M4) at (-1.000000,-2.000000) {\mobile};
\node[mobile] at (M4.south east) {$M_4$};
\draw[speed] (M4.center) -- ($ (M4.center) + (-0.500000,0.500000) $);
\draw[camino] (M0) -- (0.000000,2.000000);
\draw[interceptor] (0.000000,0.000000) -- (0.000000,2.000000);
\node[interceptor] at (0.000000,2.000000) {\mobile};
\node[caught] at (M0) {\mobile};
\draw[camino] (M1) -- (0.000000,2.000000);
\draw[interceptor] (0.000000,2.000000) -- (0.000000,2.000000);
\node[interceptor] at (0.000000,2.000000) {\mobile};
\node[caught] at (M1) {\mobile};
\draw[camino] (M4) -- (-2.384821,-0.615179);
\draw[interceptor] (0.000000,2.000000) -- (-2.384821,-0.615179);
\node[interceptor] at (-2.384821,-0.615179) {\mobile};
\node[caught] at (M4) {\mobile};
\draw[camino] (M3) -- (8.509197,5.509197);
\draw[interceptor] (-2.384821,-0.615179) -- (8.509197,5.509197);
\node[interceptor] at (8.509197,5.509197) {\mobile};
\node[caught] at (M3) {\mobile};
\draw[camino] (M2) -- (14.344612,8.703384);
\draw[interceptor] (8.509197,5.509197) -- (14.344612,8.703384);
\node[interceptor] at (14.344612,8.703384) {\mobile};
\node[caught] at (M2) {\mobile};
\draw[interceptor] (14.344612,8.703384) node[anchor=south east] {$t=12.345 \text{ u.t.}$};

      \end{tikzpicture}}
    \end{center}
    \caption{Heuristique $H_0$: Test \no1}
    \label{fig:H0_1}
  \end{figure}

  \begin{table}[H]
    \centering
    \begin{tabular}{|c|c|c|}
  \hline\textbf{\No mobile} & \textbf{Position interception} & \textbf{Date interception (u.t.)} \\ \hline 
  0	& $\coord{0.000}{2.000}$	 & $1.0000$ \\ \hline
  1	& $\coord{0.000}{2.000}$	 & $1.0000$ \\ \hline
  4	& $\coord{-2.385}{-0.615}$	 & $2.7696$ \\ \hline
  3	& $\coord{8.509}{5.509}$	 & $9.0184$ \\ \hline
  2	& $\coord{14.345}{8.703}$	 & $12.3446$ \\ \hline
\end{tabular}

    \caption{Heuristique $H_0$: Résultats test \no1}
    \label{tab:H0_1}
  \end{table}

  \begin{figure}[H]
    \begin{center}
      \boxed{
      \begin{tikzpicture}[scale=0.7]
        \draw[grided,step=1.0,thin] (-9.000000,-2.000000) grid (7.323262,6.596979);
\draw (-9.000000,0) -- coordinate (x axis mid) (7.323262,0);
\draw (0,-2.000000) -- coordinate (y axis mid) (0,6.596979);
\foreach \x in {-9,...,7}
  \draw (\x,1pt) -- (\x,-3pt) node[anchor=north] {\x};
\foreach \y in {-2,...,6}
  \draw (1pt,\y) -- (-3pt,\y) node[anchor=east] {\y};
\node[interceptor] (I0) at (0.000000,0.000000) {\interceptor};
\node[mobile,anchor=center] (M0) at (1.000000,2.000000) {\mobile};
\node[mobile] at (M0.south east) {$M_0$};
\draw[speed] (M0.center) -- ($ (M0.center) + (-1.000000,0.000000) $);
\node[mobile,anchor=center] (M1) at (0.000000,3.000000) {\mobile};
\node[mobile] at (M1.south east) {$M_1$};
\draw[speed] (M1.center) -- ($ (M1.center) + (0.000000,-1.000000) $);
\node[mobile,anchor=center] (M2) at (2.000000,5.000000) {\mobile};
\node[mobile] at (M2.south east) {$M_2$};
\draw[speed] (M2.center) -- ($ (M2.center) + (1.000000,0.300000) $);
\node[mobile,anchor=center] (M3) at (4.000000,1.000000) {\mobile};
\node[mobile] at (M3.south east) {$M_3$};
\draw[speed] (M3.center) -- ($ (M3.center) + (0.500000,0.500000) $);
\node[mobile,anchor=center] (M4) at (-1.000000,-2.000000) {\mobile};
\node[mobile] at (M4.south east) {$M_4$};
\draw[speed] (M4.center) -- ($ (M4.center) + (-0.500000,0.500000) $);
\draw[camino] (M0) -- (0.000000,2.000000);
\draw[interceptor] (0.000000,0.000000) -- (0.000000,2.000000);
\node[interceptor] at (0.000000,2.000000) {\mobile};
\node[caught] at (M0) {\mobile};
\draw[camino] (M1) -- (0.000000,2.000000);
\draw[interceptor] (0.000000,2.000000) -- (0.000000,2.000000);
\node[interceptor] at (0.000000,2.000000) {\mobile};
\node[caught] at (M1) {\mobile};
\draw[camino] (M2) -- (7.323262,6.596979);
\draw[interceptor] (0.000000,2.000000) -- (7.323262,6.596979);
\node[interceptor] at (7.323262,6.596979) {\mobile};
\node[caught] at (M2) {\mobile};
\draw[camino] (M3) -- (7.248936,4.248936);
\draw[interceptor] (7.323262,6.596979) -- (7.248936,4.248936);
\node[interceptor] at (7.248936,4.248936) {\mobile};
\node[caught] at (M3) {\mobile};
\draw[camino] (M4) -- (-8.089226,5.089226);
\draw[interceptor] (7.248936,4.248936) -- (-8.089226,5.089226);
\node[interceptor] at (-8.089226,5.089226) {\mobile};
\node[caught] at (M4) {\mobile};
\draw[interceptor] (-8.089226,5.089226) node[anchor=south west] {$t=14.178 \text{ u.t.}$};

      \end{tikzpicture}}
    \end{center}
    \caption{Heuristique $H_1$: Test \no1}
    \label{fig:H1_1}
  \end{figure}

  \begin{table}[H]
    \centering
    \begin{tabular}{|c|c|c|}
  \hline\textbf{\No mobile} & \textbf{Position interception} & \textbf{Date interception (u.t.)} \\ \hline 
  0	& $\coord{0.000}{2.000}$	 & $1.0000$ \\ \hline
  1	& $\coord{0.000}{2.000}$	 & $1.0000$ \\ \hline
  2	& $\coord{7.323}{6.597}$	 & $5.3233$ \\ \hline
  3	& $\coord{7.249}{4.249}$	 & $6.4979$ \\ \hline
  4	& $\coord{-8.089}{5.089}$	 & $14.1785$ \\ \hline
\end{tabular}

    \caption{Heuristique $H_1$: Résultats test \no1}
    \label{tab:H1_1}
  \end{table}

  \begin{figure}[H]
    \centering
    \boxed{
    \begin{tikzpicture}[yscale=0.5]
      \draw (0,0) -- coordinate (x axis mid) (5,0);
\foreach \x in {0,...,5}
  \draw (\x,1pt) -- (\x,-3pt) node[anchor=north] {\x};
\draw (0,0) -- coordinate (y axis mid) (0,13.000000);
\node[h0] at (1,7.500000) {$H_0$};
\node[h1] at (1,5.500000) {$H_1$};
\foreach \y in {0,1,...,13}
  \draw (1pt,\y) -- (-3pt,\y) node[anchor=east] {\y};
\draw (2.500000,-2) node[anchor=north] {Nombre de mobiles interceptés};
\draw (-0.75,6.500000) node[rotate=90,anchor=south] {Temps nécessaire (u.t)};
\draw[grided,step=1.0,thin] (0,0) grid (5,13.000000);
\node[h0] at (0,0) {\cross};
\draw[h0] (0,0.000000) -- (1,1.000000);
\node[h0] at(1,1.000000) {\cross};
\draw[h0] (1,1.000000) -- (2,1.000000);
\node[h0] at(2,1.000000) {\cross};
\draw[h0] (2,1.000000) -- (3,2.769642);
\node[h0] at(3,2.769642) {\cross};
\draw[h0] (3,2.769642) -- (4,9.018394);
\node[h0] at(4,9.018394) {\cross};
\draw[h0] (4,9.018394) -- (5,12.344612);
\node[h0] at(5,12.344612) {\cross};
\draw[grided,step=1.0,thin] (0,0) grid (5,15.000000);
\node[h1] at (0,0) {\cross};
\draw[h1] (0,0.000000) -- (1,1.000000);
\node[h1] at(1,1.000000) {\cross};
\draw[h1] (1,1.000000) -- (2,1.000000);
\node[h1] at(2,1.000000) {\cross};
\draw[h1] (2,1.000000) -- (3,5.323262);
\node[h1] at(3,5.323262) {\cross};
\draw[h1] (3,5.323262) -- (4,6.497872);
\node[h1] at(4,6.497872) {\cross};
\draw[h1] (4,6.497872) -- (5,14.178453);
\node[h1] at(5,14.178453) {\cross};

    \end{tikzpicture}}
    \caption{Comparaison de $H_0$ et de $H_1$: test \no1}
    \label{fig:comp_1}
  \end{figure}

\section{Test \no2: Résultats identiques et mobile non-intercepté}
  \begin{listing}[H]
    \textfile{../tests/test_2/test_2.data}
    \caption{test\_2.data}
  \end{listing}

  \begin{figure}[H]
    \begin{center}
      \boxed{
      \begin{tikzpicture}[scale=1]
        \draw[grided,step=1.0,thin] (-2.000000,-1.000000) grid (5.797277,5.000000);
\draw (-2.000000,0) -- coordinate (x axis mid) (5.797277,0);
\draw (0,-1.000000) -- coordinate (y axis mid) (0,5.000000);
\foreach \x in {-2,...,5}
  \draw (\x,1pt) -- (\x,-3pt) node[anchor=north] {\x};
\foreach \y in {-1,...,5}
  \draw (1pt,\y) -- (-3pt,\y) node[anchor=east] {\y};
\node[interceptor] (I0) at (0.000000,0.000000) {\interceptor};
\node[mobile,anchor=center] (M0) at (2.000000,2.000000) {\mobile};
\node[mobile] at (M0.south east) {$M_0$};
\draw[speed] (M0.center) -- ($ (M0.center) + (-1.500000,0.000000) $);
\node[mobile,anchor=center] (M1) at (-1.000000,-1.000000) {\mobile};
\node[mobile] at (M1.south east) {$M_1$};
\draw[speed] (M1.center) -- ($ (M1.center) + (0.000000,2.000000) $);
\node[mobile,anchor=center] (M2) at (4.000000,5.000000) {\mobile};
\node[mobile] at (M2.south east) {$M_2$};
\draw[speed] (M2.center) -- ($ (M2.center) + (-1.000000,-0.300000) $);
\node[mobile,anchor=center] (M3) at (2.000000,4.000000) {\mobile};
\node[mobile] at (M3.south east) {$M_3$};
\draw[speed] (M3.center) -- ($ (M3.center) + (0.600000,-0.690000) $);
\node[mobile,anchor=center] (M4) at (-2.000000,-1.000000) {\mobile};
\node[mobile] at (M4.south east) {$M_4$};
\draw[speed] (M4.center) -- ($ (M4.center) + (1.000000,6.000000) $);
\draw[camino] (M0) -- (0.460716,2.000000);
\draw[interceptor] (0.000000,0.000000) -- (0.460716,2.000000);
\node[interceptor] at (0.460716,2.000000) {\mobile};
\node[caught] at (M0) {\mobile};
\draw[camino] (M1) -- (-1.000000,2.652004);
\draw[interceptor] (0.460716,2.000000) -- (-1.000000,2.652004);
\node[interceptor] at (-1.000000,2.652004) {\mobile};
\node[caught] at (M1) {\mobile};
\draw[camino] (M2) -- (0.959412,4.087824);
\draw[interceptor] (-1.000000,2.652004) -- (0.959412,4.087824);
\node[interceptor] at (0.959412,4.087824) {\mobile};
\node[caught] at (M2) {\mobile};
\draw[camino] (M3) -- (5.797277,-0.366868);
\draw[interceptor] (0.959412,4.087824) -- (5.797277,-0.366868);
\node[interceptor] at (5.797277,-0.366868) {\mobile};
\node[caught] at (M3) {\mobile};
\draw[interceptor] (5.797277,-0.366868) node[anchor=south east] {$t=6.329 \text{ u.t.}$};

      \end{tikzpicture}}
    \end{center}
    \caption{Heuristique $H_0$: Test \no2}
    \label{fig:H0_2}
  \end{figure}

  \begin{table}[H]
    \centering
    \begin{tabular}{|c|c|c|}
  \hline\textbf{\No mobile} & \textbf{Position interception} & \textbf{Date interception (u.t.)} \\ \hline 
  0	& $\coord{0.461}{2.000}$	 & $1.0262$ \\ \hline
  1	& $\coord{-1.000}{2.652}$	 & $1.8260$ \\ \hline
  2	& $\coord{0.959}{4.088}$	 & $3.0406$ \\ \hline
  3	& $\coord{5.797}{-0.367}$	 & $6.3288$ \\ \hline
\end{tabular}

    \caption{Heuristique $H_0$: Résultats test \no2}
    \label{tab:H0_2}
  \end{table}

  \begin{figure}[H]
    \begin{center}
      \boxed{
      \begin{tikzpicture}[scale=1]
        \draw[grided,step=1.0,thin] (-2.000000,-1.000000) grid (5.797277,5.000000);
\draw (-2.000000,0) -- coordinate (x axis mid) (5.797277,0);
\draw (0,-1.000000) -- coordinate (y axis mid) (0,5.000000);
\foreach \x in {-2,...,5}
  \draw (\x,1pt) -- (\x,-3pt) node[anchor=north] {\x};
\foreach \y in {-1,...,5}
  \draw (1pt,\y) -- (-3pt,\y) node[anchor=east] {\y};
\node[interceptor] (I0) at (0.000000,0.000000) {\interceptor};
\node[mobile,anchor=center] (M0) at (2.000000,2.000000) {\mobile};
\node[mobile] at (M0.south east) {$M_0$};
\draw[speed] (M0.center) -- ($ (M0.center) + (-1.500000,0.000000) $);
\node[mobile,anchor=center] (M1) at (-1.000000,-1.000000) {\mobile};
\node[mobile] at (M1.south east) {$M_1$};
\draw[speed] (M1.center) -- ($ (M1.center) + (0.000000,2.000000) $);
\node[mobile,anchor=center] (M2) at (4.000000,5.000000) {\mobile};
\node[mobile] at (M2.south east) {$M_2$};
\draw[speed] (M2.center) -- ($ (M2.center) + (-1.000000,-0.300000) $);
\node[mobile,anchor=center] (M3) at (2.000000,4.000000) {\mobile};
\node[mobile] at (M3.south east) {$M_3$};
\draw[speed] (M3.center) -- ($ (M3.center) + (0.600000,-0.690000) $);
\node[mobile,anchor=center] (M4) at (-2.000000,-1.000000) {\mobile};
\node[mobile] at (M4.south east) {$M_4$};
\draw[speed] (M4.center) -- ($ (M4.center) + (1.000000,6.000000) $);
\draw[camino] (M0) -- (0.460716,2.000000);
\draw[interceptor] (0.000000,0.000000) -- (0.460716,2.000000);
\node[interceptor] at (0.460716,2.000000) {\mobile};
\node[caught] at (M0) {\mobile};
\draw[camino] (M1) -- (-1.000000,2.652004);
\draw[interceptor] (0.460716,2.000000) -- (-1.000000,2.652004);
\node[interceptor] at (-1.000000,2.652004) {\mobile};
\node[caught] at (M1) {\mobile};
\draw[camino] (M2) -- (0.959412,4.087824);
\draw[interceptor] (-1.000000,2.652004) -- (0.959412,4.087824);
\node[interceptor] at (0.959412,4.087824) {\mobile};
\node[caught] at (M2) {\mobile};
\draw[camino] (M3) -- (5.797277,-0.366868);
\draw[interceptor] (0.959412,4.087824) -- (5.797277,-0.366868);
\node[interceptor] at (5.797277,-0.366868) {\mobile};
\node[caught] at (M3) {\mobile};
\draw[interceptor] (5.797277,-0.366868) node[anchor=south east] {$t=6.329 \text{ u.t.}$};

      \end{tikzpicture}}
    \end{center}
    \caption{Heuristique $H_1$: Test \no2}
    \label{fig:H1_2}
  \end{figure}

  \begin{table}[H]
    \centering
    \begin{tabular}{|c|c|c|}
  \hline\textbf{\No mobile} & \textbf{Position interception} & \textbf{Date interception (u.t.)} \\ \hline 
  0	& $\coord{0.461}{2.000}$	 & $1.0262$ \\ \hline
  1	& $\coord{-1.000}{2.652}$	 & $1.8260$ \\ \hline
  2	& $\coord{0.959}{4.088}$	 & $3.0406$ \\ \hline
  3	& $\coord{5.797}{-0.367}$	 & $6.3288$ \\ \hline
\end{tabular}

    \caption{Heuristique $H_1$: Résultats test \no2}
    \label{tab:H1_2}
  \end{table}

  \begin{figure}[H]
    \centering
    \boxed{
    \begin{tikzpicture}[yscale=0.5]
      \draw (0,0) -- coordinate (x axis mid) (5,0);
\foreach \x in {0,...,5}
  \draw (\x,1pt) -- (\x,-3pt) node[anchor=north] {\x};
\draw (0,0) -- coordinate (y axis mid) (0,7.000000);
\foreach \y in {0,1,...,7}
  \draw (1pt,\y) -- (-3pt,\y) node[anchor=east] {\y};
\draw (2.500000,-2) node[anchor=north] {Nombre de mobiles interceptés};
\draw (-0.75,3.500000) node[rotate=90,anchor=south] {Temps nécessaire (u.t)};
\draw[grided,step=1.0,thin] (0,0) grid (5,7.000000);
\node[h0] at (0,0) {\cross};
\draw[h0] (0,0.000000) -- (1,1.026189);
\node[h0] at(1,1.026189) {\cross};
\draw[h0] (1,1.026189) -- (2,1.826002);
\node[h0] at(2,1.826002) {\cross};
\draw[h0] (2,1.826002) -- (3,3.040588);
\node[h0] at(3,3.040588) {\cross};
\draw[h0] (3,3.040588) -- (4,6.328794);
\node[h0] at(4,6.328794) {\cross};
\draw[grided,step=1.0,thin] (0,0) grid (5,7.000000);
\node[h1] at (0,0) {\cross};
\draw[h1] (0,0.000000) -- (1,1.026189);
\node[h1] at(1,1.026189) {\cross};
\draw[h1] (1,1.026189) -- (2,1.826002);
\node[h1] at(2,1.826002) {\cross};
\draw[h1] (2,1.826002) -- (3,3.040588);
\node[h1] at(3,3.040588) {\cross};
\draw[h1] (3,3.040588) -- (4,6.328794);
\node[h1] at(4,6.328794) {\cross};

    \end{tikzpicture}}
    \caption{Comparaison de $H_0$ et de $H_1$: test \no2}
    \label{fig:comp_2}
  \end{figure}


\section{Test \no3: Mobiles positionnés aléatoirement}
  \begin{listing}[H]
    \textfile{../tests/test_3/test_3.data}
    \caption{test\_3.data}
  \end{listing}

  \begin{figure}[H]
    \begin{center}
      \boxed{
      \begin{tikzpicture}[scale=0.5]
        \draw[grided,step=1.0,thin] (-15.000000,-9.000000) grid (8.360891,9.000000);
\draw (-15.000000,0) -- coordinate (x axis mid) (8.360891,0);
\draw (0,-9.000000) -- coordinate (y axis mid) (0,9.000000);
\foreach \x in {-15,...,8}
  \draw (\x,1pt) -- (\x,-3pt) node[anchor=north] {\x};
\foreach \y in {-9,...,9}
  \draw (1pt,\y) -- (-3pt,\y) node[anchor=east] {\y};
\node[interceptor] (I0) at (2.000000,9.000000) {\interceptor};
\node[mobile,anchor=center] (M0) at (7.140297,8.876400) {\mobile};
\node[mobile] at (M0.south east) {$M_0$};
\node[mobile,anchor=center] (M1) at (4.767767,5.291864) {\mobile};
\node[mobile] at (M1.south east) {$M_1$};
\draw[speed] (M1.center) -- ($ (M1.center) + (-0.811979,-0.290230) $);
\node[mobile,anchor=center] (M2) at (-7.277143,-4.215198) {\mobile};
\node[mobile] at (M2.south east) {$M_2$};
\draw[speed] (M2.center) -- ($ (M2.center) + (0.458222,0.901331) $);
\node[mobile,anchor=center] (M3) at (-9.317039,-8.399123) {\mobile};
\node[mobile] at (M3.south east) {$M_3$};
\draw[speed] (M3.center) -- ($ (M3.center) + (-0.209148,0.404874) $);
\node[mobile,anchor=center] (M4) at (8.360891,5.869070) {\mobile};
\node[mobile] at (M4.south east) {$M_4$};
\draw[speed] (M4.center) -- ($ (M4.center) + (-0.505128,0.038969) $);
\node[mobile,anchor=center] (M5) at (2.371453,-8.866265) {\mobile};
\node[mobile] at (M5.south east) {$M_5$};
\draw[speed] (M5.center) -- ($ (M5.center) + (-0.225212,0.200970) $);
\draw[camino] (M1) -- (2.957371,4.644764);
\draw[interceptor] (2.000000,9.000000) -- (2.957371,4.644764);
\node[interceptor] at (2.957371,4.644764) {\mobile};
\node[caught] at (M1) {\mobile};
\draw[camino] (M4) -- (6.317120,6.026740);
\draw[interceptor] (2.957371,4.644764) -- (6.317120,6.026740);
\node[interceptor] at (6.317120,6.026740) {\mobile};
\node[caught] at (M4) {\mobile};
\draw[interceptor] (6.317120,6.026740) -- (7.140297,8.876400);
\node[interceptor] at (7.140297,8.876400) {\mobile};
\node[caught] at (M0) {\mobile};
\draw[camino] (M2) -- (-2.411295,5.356014);
\draw[interceptor] (7.140297,8.876400) -- (-2.411295,5.356014);
\node[interceptor] at (-2.411295,5.356014) {\mobile};
\node[caught] at (M2) {\mobile};
\draw[camino] (M5) -- (-1.263106,-5.622933);
\draw[interceptor] (-2.411295,5.356014) -- (-1.263106,-5.622933);
\node[interceptor] at (-1.263106,-5.622933) {\mobile};
\node[caught] at (M5) {\mobile};
\draw[camino] (M3) -- (-14.218112,1.088498);
\draw[interceptor] (-1.263106,-5.622933) -- (-14.218112,1.088498);
\node[interceptor] at (-14.218112,1.088498) {\mobile};
\node[caught] at (M3) {\mobile};
\draw[interceptor] (-14.218112,1.088498) node[anchor=south west] {$t=23.434 \text{ u.t.}$};

      \end{tikzpicture}}
    \end{center}
    \caption{Heuristique $H_0$: Test \no3}
    \label{fig:H0_3}
  \end{figure}

  \begin{table}[H]
    \centering
    \begin{tabular}{|c|c|c|}
  \hline\textbf{\No mobile} & \textbf{Position interception} & \textbf{Date interception (u.t.)} \\ \hline 
  1	& $\coord{2.957}{4.645}$	 & $2.2296$ \\ \hline
  4	& $\coord{6.317}{6.027}$	 & $4.0460$ \\ \hline
  0	& $\coord{7.140}{8.876}$	 & $5.5291$ \\ \hline
  2	& $\coord{-2.411}{5.356}$	 & $10.6190$ \\ \hline
  5	& $\coord{-1.263}{-5.623}$	 & $16.1384$ \\ \hline
  3	& $\coord{-14.218}{1.088}$	 & $23.4335$ \\ \hline
\end{tabular}

    \caption{Heuristique $H_0$: Résultats test \no3}
    \label{tab:H0_3}
  \end{table}

  \begin{figure}[H]
    \begin{center}
      \boxed{
      \begin{tikzpicture}[scale=0.5]
        \draw[grided,step=1.0,thin] (-13.000000,-9.000000) grid (8.360891,9.000000);
\draw (-13.000000,0) -- coordinate (x axis mid) (8.360891,0);
\draw (0,-9.000000) -- coordinate (y axis mid) (0,9.000000);
\foreach \x in {-13,...,8}
  \draw (\x,1pt) -- (\x,-3pt) node[anchor=north] {\x};
\foreach \y in {-9,...,9}
  \draw (1pt,\y) -- (-3pt,\y) node[anchor=east] {\y};
\node[interceptor] (I0) at (2.000000,9.000000) {\interceptor};
\node[mobile,anchor=center] (M0) at (7.140297,8.876400) {\mobile};
\node[mobile] at (M0.south east) {$M_0$};
\node[mobile,anchor=center] (M1) at (4.767767,5.291864) {\mobile};
\node[mobile] at (M1.south east) {$M_1$};
\draw[speed] (M1.center) -- ($ (M1.center) + (-0.811979,-0.290230) $);
\node[mobile,anchor=center] (M2) at (-7.277143,-4.215198) {\mobile};
\node[mobile] at (M2.south east) {$M_2$};
\draw[speed] (M2.center) -- ($ (M2.center) + (0.458222,0.901331) $);
\node[mobile,anchor=center] (M3) at (-9.317039,-8.399123) {\mobile};
\node[mobile] at (M3.south east) {$M_3$};
\draw[speed] (M3.center) -- ($ (M3.center) + (-0.209148,0.404874) $);
\node[mobile,anchor=center] (M4) at (8.360891,5.869070) {\mobile};
\node[mobile] at (M4.south east) {$M_4$};
\draw[speed] (M4.center) -- ($ (M4.center) + (-0.505128,0.038969) $);
\node[mobile,anchor=center] (M5) at (2.371453,-8.866265) {\mobile};
\node[mobile] at (M5.south east) {$M_5$};
\draw[speed] (M5.center) -- ($ (M5.center) + (-0.225212,0.200970) $);
\draw[interceptor] (2.000000,9.000000) -- (7.140297,8.876400);
\node[interceptor] at (7.140297,8.876400) {\mobile};
\node[caught] at (M0) {\mobile};
\draw[camino] (M1) -- (-1.586399,3.020660);
\draw[interceptor] (7.140297,8.876400) -- (-1.586399,3.020660);
\node[interceptor] at (-1.586399,3.020660) {\mobile};
\node[caught] at (M1) {\mobile};
\draw[camino] (M2) -- (-3.277717,3.651746);
\draw[interceptor] (-1.586399,3.020660) -- (-3.277717,3.651746);
\node[interceptor] at (-3.277717,3.651746) {\mobile};
\node[caught] at (M2) {\mobile};
\draw[camino] (M3) -- (-12.292086,-2.639952);
\draw[interceptor] (-3.277717,3.651746) -- (-12.292086,-2.639952);
\node[interceptor] at (-12.292086,-2.639952) {\mobile};
\node[caught] at (M3) {\mobile};
\draw[camino] (M4) -- (-2.280671,6.690032);
\draw[interceptor] (-12.292086,-2.639952) -- (-2.280671,6.690032);
\node[interceptor] at (-2.280671,6.690032) {\mobile};
\node[caught] at (M4) {\mobile};
\draw[camino] (M5) -- (-3.539516,-3.591557);
\draw[interceptor] (-2.280671,6.690032) -- (-3.539516,-3.591557);
\node[interceptor] at (-3.539516,-3.591557) {\mobile};
\node[caught] at (M5) {\mobile};
\draw[interceptor] (-3.539516,-3.591557) node[anchor=south east] {$t=26.246 \text{ u.t.}$};

      \end{tikzpicture}}
    \end{center}
    \caption{Heuristique $H_1$: Test \no3}
    \label{fig:H1_3}
  \end{figure}

  \begin{table}[H]
    \centering
    \begin{tabular}{|c|c|c|}
  \hline\textbf{\No mobile} & \textbf{Position interception} & \textbf{Date interception (u.t.)} \\ \hline 
  0	& $\coord{7.140}{8.876}$	 & $2.5709$ \\ \hline
  1	& $\coord{-1.586}{3.021}$	 & $7.8255$ \\ \hline
  2	& $\coord{-3.278}{3.652}$	 & $8.7281$ \\ \hline
  3	& $\coord{-12.292}{-2.640}$	 & $14.2246$ \\ \hline
  4	& $\coord{-2.281}{6.690}$	 & $21.0671$ \\ \hline
  5	& $\coord{-3.540}{-3.592}$	 & $26.2462$ \\ \hline
\end{tabular}

    \caption{Heuristique $H_1$: Résultats test \no3}
    \label{tab:H1_3}
  \end{table}

  \begin{figure}[H]
    \centering
    \boxed{
    \begin{tikzpicture}[yscale=0.2]
      \draw (0,0) -- coordinate (x axis mid) (6,0);
\foreach \x in {0,...,6}
  \draw (\x,1pt) -- (\x,-3pt) node[anchor=north] {\x};
\draw (0,0) -- coordinate (y axis mid) (0,24.000000);
\foreach \y in {0,2,...,24}
  \draw (1pt,\y) -- (-3pt,\y) node[anchor=east] {\y};
\draw (3.000000,-2) node[anchor=north] {Nombre de mobiles interceptés};
\draw (-0.75,12.000000) node[rotate=90,anchor=south] {Temps nécessaire (u.t)};
\draw[grided,step=1.0,thin] (0,0) grid (6,24.000000);
\node[h0] at (0,0) {\cross};
\draw[h0] (0,0.000000) -- (1,2.229610);
\node[h0] at(1,2.229610) {\cross};
\draw[h0] (1,2.229610) -- (2,4.046046);
\node[h0] at(2,4.046046) {\cross};
\draw[h0] (2,4.046046) -- (3,5.529133);
\node[h0] at(3,5.529133) {\cross};
\draw[h0] (3,5.529133) -- (4,10.618976);
\node[h0] at(4,10.618976) {\cross};
\draw[h0] (4,10.618976) -- (5,16.138388);
\node[h0] at(5,16.138388) {\cross};
\draw[h0] (5,16.138388) -- (6,23.433514);
\node[h0] at(6,23.433514) {\cross};
\draw[grided,step=1.0,thin] (0,0) grid (6,27.000000);
\node[h1] at (0,0) {\cross};
\draw[h1] (0,0.000000) -- (1,2.570891);
\node[h1] at(1,2.570891) {\cross};
\draw[h1] (1,2.570891) -- (2,7.825530);
\node[h1] at(2,7.825530) {\cross};
\draw[h1] (2,7.825530) -- (3,8.728141);
\node[h1] at(3,8.728141) {\cross};
\draw[h1] (3,8.728141) -- (4,14.224601);
\node[h1] at(4,14.224601) {\cross};
\draw[h1] (4,14.224601) -- (5,21.067061);
\node[h1] at(5,21.067061) {\cross};
\draw[h1] (5,21.067061) -- (6,26.246245);
\node[h1] at(6,26.246245) {\cross};

    \end{tikzpicture}}
    \caption{Comparaison de $H_0$ et de $H_1$: test \no3}
    \label{fig:comp_3}
  \end{figure}

  \section{Test \no4: Heuristique $H_1$ plus rapide}
  \begin{listing}[H]
    \textfile{../tests/test_4/test_4.data}
    \caption{test\_4.data}
  \end{listing}

  \begin{figure}[H]
    \begin{center}
      \boxed{
      \begin{tikzpicture}[scale=0.5]
        \draw[grided,step=1.0,thin] (-2.000000,-8.000000) grid (14.363306,9.297633);
\draw (-2.000000,0) -- coordinate (x axis mid) (14.363306,0);
\draw (0,-8.000000) -- coordinate (y axis mid) (0,9.297633);
\foreach \x in {-2,...,14}
  \draw (\x,1pt) -- (\x,-3pt) node[anchor=north] {\x};
\foreach \y in {-8,...,9}
  \draw (1pt,\y) -- (-3pt,\y) node[anchor=east] {\y};
\node[interceptor] (I0) at (0.000000,0.000000) {\interceptor};
\node[mobile,anchor=center] (M0) at (2.000000,2.000000) {\mobile};
\node[mobile] at (M0.south east) {$M_0$};
\draw[speed] (M0.center) -- ($ (M0.center) + (-1.000000,0.000000) $);
\node[mobile,anchor=center] (M1) at (-1.000000,-1.000000) {\mobile};
\node[mobile] at (M1.south east) {$M_1$};
\draw[speed] (M1.center) -- ($ (M1.center) + (0.000000,2.000000) $);
\node[mobile,anchor=center] (M2) at (4.000000,5.000000) {\mobile};
\node[mobile] at (M2.south east) {$M_2$};
\draw[speed] (M2.center) -- ($ (M2.center) + (-1.000000,-0.300000) $);
\node[mobile,anchor=center] (M3) at (2.000000,4.000000) {\mobile};
\node[mobile] at (M3.south east) {$M_3$};
\draw[speed] (M3.center) -- ($ (M3.center) + (0.750000,-0.690000) $);
\draw[camino] (M0) -- (0.902832,2.000000);
\draw[interceptor] (0.000000,0.000000) -- (0.902832,2.000000);
\node[interceptor] at (0.902832,2.000000) {\mobile};
\node[caught] at (M0) {\mobile};
\draw[camino] (M2) -- (1.685421,4.305626);
\draw[interceptor] (0.902832,2.000000) -- (1.685421,4.305626);
\node[interceptor] at (1.685421,4.305626) {\mobile};
\node[caught] at (M2) {\mobile};
\draw[camino] (M1) -- (-1.000000,9.297633);
\draw[interceptor] (1.685421,4.305626) -- (-1.000000,9.297633);
\node[interceptor] at (-1.000000,9.297633) {\mobile};
\node[caught] at (M1) {\mobile};
\draw[camino] (M3) -- (14.363306,-7.374242);
\draw[interceptor] (-1.000000,9.297633) -- (14.363306,-7.374242);
\node[interceptor] at (14.363306,-7.374242) {\mobile};
\node[caught] at (M3) {\mobile};
\draw[interceptor] (14.363306,-7.374242) node[anchor=south east] {$t=16.484 \text{ u.t.}$};

      \end{tikzpicture}}
    \end{center}
    \caption{Heuristique $H_0$: Test \no4}
    \label{fig:H0_4}
  \end{figure}

  \begin{table}[H]
    \centering
    \begin{tabular}{|c|c|c|}
  \hline\textbf{\No mobile} & \textbf{Position interception} & \textbf{Date interception (u.t.)} \\ \hline 
  0	& $\coord{0.903}{2.000}$	 & $1.0972$ \\ \hline
  2	& $\coord{1.685}{4.306}$	 & $2.3146$ \\ \hline
  1	& $\coord{-1.000}{9.298}$	 & $5.1488$ \\ \hline
  3	& $\coord{14.363}{-7.374}$	 & $16.4844$ \\ \hline
\end{tabular}

    \caption{Heuristique $H_0$: Résultats test \no4}
    \label{tab:H0_4}
  \end{table}

  \begin{figure}[H]
    \begin{center}
      \boxed{
      \begin{tikzpicture}[scale=1]
        \draw[grided,step=1.0,thin] (-1.000000,-2.000000) grid (7.784527,5.000000);
\draw (-1.000000,0) -- coordinate (x axis mid) (7.784527,0);
\draw (0,-2.000000) -- coordinate (y axis mid) (0,5.000000);
\foreach \x in {-1,...,7}
  \draw (\x,1pt) -- (\x,-3pt) node[anchor=north] {\x};
\foreach \y in {-2,...,5}
  \draw (1pt,\y) -- (-3pt,\y) node[anchor=east] {\y};
\node[interceptor] (I0) at (0.000000,0.000000) {\interceptor};
\node[mobile,anchor=center] (M0) at (2.000000,2.000000) {\mobile};
\node[mobile] at (M0.south east) {$M_0$};
\draw[speed] (M0.center) -- ($ (M0.center) + (-1.000000,0.000000) $);
\node[mobile,anchor=center] (M1) at (-1.000000,-1.000000) {\mobile};
\node[mobile] at (M1.south east) {$M_1$};
\draw[speed] (M1.center) -- ($ (M1.center) + (0.000000,2.000000) $);
\node[mobile,anchor=center] (M2) at (4.000000,5.000000) {\mobile};
\node[mobile] at (M2.south east) {$M_2$};
\draw[speed] (M2.center) -- ($ (M2.center) + (-1.000000,-0.300000) $);
\node[mobile,anchor=center] (M3) at (2.000000,4.000000) {\mobile};
\node[mobile] at (M3.south east) {$M_3$};
\draw[speed] (M3.center) -- ($ (M3.center) + (0.750000,-0.690000) $);
\draw[camino] (M0) -- (0.902832,2.000000);
\draw[interceptor] (0.000000,0.000000) -- (0.902832,2.000000);
\node[interceptor] at (0.902832,2.000000) {\mobile};
\node[caught] at (M0) {\mobile};
\draw[camino] (M1) -- (-1.000000,3.844238);
\draw[interceptor] (0.902832,2.000000) -- (-1.000000,3.844238);
\node[interceptor] at (-1.000000,3.844238) {\mobile};
\node[caught] at (M1) {\mobile};
\draw[camino] (M2) -- (0.715771,4.014731);
\draw[interceptor] (-1.000000,3.844238) -- (0.715771,4.014731);
\node[interceptor] at (0.715771,4.014731) {\mobile};
\node[caught] at (M2) {\mobile};
\draw[camino] (M3) -- (7.784527,-1.321765);
\draw[interceptor] (0.715771,4.014731) -- (7.784527,-1.321765);
\node[interceptor] at (7.784527,-1.321765) {\mobile};
\node[caught] at (M3) {\mobile};
\draw[interceptor] (7.784527,-1.321765) node[anchor=south east] {$t=7.713 \text{ u.t.}$};

      \end{tikzpicture}}
    \end{center}
    \caption{Heuristique $H_1$: Test \no4}
    \label{fig:H1_4}
  \end{figure}

  \begin{table}[H]
    \centering
    \begin{tabular}{|c|c|c|}
  \hline\textbf{\No mobile} & \textbf{Position interception} & \textbf{Date interception (u.t.)} \\ \hline 
  0	& $\coord{0.903}{2.000}$	 & $1.0972$ \\ \hline
  1	& $\coord{-1.000}{3.844}$	 & $2.4221$ \\ \hline
  2	& $\coord{0.716}{4.015}$	 & $3.2842$ \\ \hline
  3	& $\coord{7.785}{-1.322}$	 & $7.7127$ \\ \hline
\end{tabular}

    \caption{Heuristique $H_1$: Résultats test \no4}
    \label{tab:H1_4}
  \end{table}

  \begin{figure}[H]
    \centering
    \boxed{
    \begin{tikzpicture}[yscale=0.35]
      \draw (0,0) -- coordinate (x axis mid) (4,0);
\foreach \x in {0,...,4}
  \draw (\x,1pt) -- (\x,-3pt) node[anchor=north] {\x};
\draw (0,0) -- coordinate (y axis mid) (0,17.000000);
\node[h0] at (1,9.500000) {$H_0$};
\node[h1] at (1,7.500000) {$H_1$};
\foreach \y in {0,2,...,17}
  \draw (1pt,\y) -- (-3pt,\y) node[anchor=east] {\y};
\draw (2.000000,-2) node[anchor=north] {Nombre de mobiles interceptés};
\draw (-0.75,8.500000) node[rotate=90,anchor=south] {Temps nécessaire (u.t)};
\draw[grided,step=1.0,thin] (0,0) grid (4,17.000000);
\node[h0] at (0,0) {\cross};
\draw[h0] (0,0.000000) -- (1,1.097168);
\node[h0] at(1,1.097168) {\cross};
\draw[h0] (1,1.097168) -- (2,2.314579);
\node[h0] at(2,2.314579) {\cross};
\draw[h0] (2,2.314579) -- (3,5.148816);
\node[h0] at(3,5.148816) {\cross};
\draw[h0] (3,5.148816) -- (4,16.484408);
\node[h0] at(4,16.484408) {\cross};
\draw[grided,step=1.0,thin] (0,0) grid (4,8.000000);
\node[h1] at (0,0) {\cross};
\draw[h1] (0,0.000000) -- (1,1.097168);
\node[h1] at(1,1.097168) {\cross};
\draw[h1] (1,1.097168) -- (2,2.422119);
\node[h1] at(2,2.422119) {\cross};
\draw[h1] (2,2.422119) -- (3,3.284229);
\node[h1] at(3,3.284229) {\cross};
\draw[h1] (3,3.284229) -- (4,7.712703);
\node[h1] at(4,7.712703) {\cross};

    \end{tikzpicture}}
    \caption{Comparaison de $H_0$ et de $H_1$: test \no4}
    \label{fig:comp_4}
  \end{figure}
