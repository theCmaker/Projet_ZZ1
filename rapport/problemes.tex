\section{Calcul d'interception}

On modélise le déplacement de l'intercepteur par la fonction suivante : 
\[
\vect{i}(t,\alpha) = 
\left(
\begin{array}{c}
 x_1 + t \cdot v_1 \cdot \cos(\alpha)\\
 y_1 + t \cdot v_1 \cdot \sin(\alpha)
\end{array}
\right)
\]
avec $t \in \R^+$ et $\alpha \in \icc{-\pi}{\pi}$.

On modélise de même la même manière le déplacement du mobile :
\[
\vect{m}(t) = 
\left(
\begin{array}{c}
 x_0 + t \cdot v^{x}_0\\
 y_0 + t \cdot v^{y}_0
\end{array}
\right)
\]
avec $t \in \R^+$.

On doit donc résoudre le système d'équations suivant afin de calculer le temps d'interception d'un mobile:

\[
\left\{
\begin{array}{r c l}
x_1 + t \cdot v_1 \cdot \cos(\alpha) &=& x_0 + t \cdot v^{x}_0\\
y_1 + t \cdot v_1 \cdot \sin(\alpha) &=& y_0 + t \cdot v^{y}_0
\end{array}
\right.
\]

La valeur est donnée par la résolution de l'équation $a \cdot \cos(\alpha)+b \cdot \sin(\alpha) = c$ avec:
\[
\left\{
\begin{array}{r c l}
a &=& y_0 - y_1\\
b &=& x_1 - x_0\\
c &=& \displaystyle \frac{a \cdot v^{x}_0 +b \cdot v^{y}_0}{v_1}
\end{array}
\right.
\]

On obtient alors 2 possibilités pour la date t : 
\[ \frac{-b}{-v^{x}_0 + v_1 \cdot \cos(\alpha)}  \qqtext{et} \frac{a}{-v^{y}_0 + v_1 \cdot \sin(\alpha)} \]

La fonction d'interception teste alors les positions obtenues avec les deux dates et retient celle qui fonctionne et qui est minimale.

\section{Heuristique $H_0$}
	Nous proposons une première méthode heuristique $H_0$ qui intercepte successivement les mobiles qui seront interceptés le plus rapidement. 
	A chaque nouvelle interception, les temps d'interception pour atteindre les mobiles restants sont recalculés, et l'on conserve le plus faible d'entre-eux.

	Ainsi le premier mobile intercepté n'est pas nécessairement le plus proche. En effet, il suffit qu'il s'éloigne de la position initiale de l'intercepteur tandis qu'un autre, plus éloigné au départ s'en rapproche suffisamment vite pour que l'intercepteur commence par intercepter ce dernier.

	A chaque étape de recherche du temps d'interception minimal, nous conservons les paramètres d'orientation de l'intercepteur et le temps calculé.

\section{Heuristique $H_1$}
	La seconde méthode heuristique que nous proposons permet, à partir d'une séquence donnée définissant l'ordre dans lequel les mobiles doivent être interceptés, de calculer le temps qui sera nécessaire pour intercepter tous les mobiles qui sont accessibles tout en respectant l'ordre demandé.

	Cette méthode demande moins de ressources de calcul que la précédente dans la mesure où l'ordre est défini par avance: il n'y a donc pas à déterminer de temps minimal.

